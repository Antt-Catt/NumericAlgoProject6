\documentclass{article}

\usepackage[english]{babel}
\usepackage[utf8]{inputenc}
\usepackage[]{hyperref}
\usepackage{graphicx, float}
\usepackage{subfig}
\usepackage{amssymb, amsmath, amsthm}

\usepackage{amsmath} 
\renewcommand*{\overrightarrow}[1]{\vbox{\halign{##\cr 
  \tiny\rightarrowfill\cr\noalign{\nointerlineskip\vskip1pt} 
  $#1\mskip2mu$\cr}}}

\begin{document}

\section*{PENDULE À N MAILLONS}

\paragraph*{1.} Dans le cas du pendule simple, il y a deux forces qui s'appliquent sur la masse \textbf{m}: le poids de la masse \overrightarrow{P} et la force de la tige \overrightarrow{T}.
D'après le théorème du moment cinétique appliqué en M par rapport à O on a
\begin{equation}
  \frac{d\overrightarrow{L_{O,M}}}{dt} = \overrightarrow{M_{O, M}(\overrightarrow{P})} + \overrightarrow{M_{O, M}(\overrightarrow{T})} \nonumber
\end{equation}
Le moment de \overrightarrow{T} est nul car \overrightarrow{OM} et \overrightarrow{T} sont colinéaires. En développant l'expression, on a alors 
$ m l^{2} \ddot \theta = - m l g \sin{\theta} $, ce qui nous donne finalement 
\begin{equation}
    \ddot \theta + \frac{g}{l} \sin{\theta}= 0 \nonumber
\end{equation}

\end{document}