\section{Pendule à $N$ maillons}

\renewcommand*{\overrightarrow}[1]{\vbox{\halign{##\cr 
  \tiny\rightarrowfill\cr\noalign{\nointerlineskip\vskip1pt} 
  $#1\mskip2mu$\cr}}}

Dans cette partie, nous allons considérer deux types de pendules à $N$ maillons (le pendule simple et le pendule double) et chercher à 
résoudre leurs équations de mouvement par les méthodes numériques implémentées dans la section~\ref{sec:sec1}.

\subsection{Pendule simple}
Dans le cas du pendule simple, il y a deux forces qui s'appliquent sur la masse $m$:
le poids de la masse \overrightarrow{P} et la force de la tige \overrightarrow{T}.
Selon le théorème du moment cinétique appliqué en $M$ par rapport à $O$, on a
$\frac{d\overrightarrow{L_{O,M}}}{dt} = \overrightarrow{M_{O, M}(\overrightarrow{P})} + \overrightarrow{M_{O, M}(\overrightarrow{T})}$.

Le moment de \overrightarrow{T} est nul, car \overrightarrow{OM} et \overrightarrow{T} sont colinéaires. En développant l'expression, on a alors 
$m l^{2} \ddot \theta = - m l g \sin{\theta} $, ce qui nous donne finalement l'équation $\ddot \theta + \frac{g}{l} \sin{\theta}= 0$.

On retrouve l'équation différentielle non linéaire du second ordre bien connue du pendule simple. 
Pour la résoudre numériquement en fonction de l'angle initial $ \theta_0 $, et de la vitesse initiale $ \dot \theta_0 $,
on peut utiliser une des méthodes implémentées auparavant, comme par exemple la méthode d'Euler.
Une fois que la fonction $ \theta(t) $ a été approchée, on peut calculer la période du pendule en prenant la différence entre 
deux maxima successifs, ce qui permet de déterminer la fréquence.

En affichant les différentes fréquences du pendule pour des valeurs de $ \theta_{0} $ variant de $ -\frac{\pi}{2} $ à $ \frac{\pi}{2} $, 
on obtient la figure~\ref{fig:frequences}.

\begin{figure}[htbp!]
	\centering
	\includegraphics[width=0.45\textwidth]{res/freq_pendule_simple.png}
	\caption{Fréquence du pendule en fonction de l'angle initial $ \theta_{0}$}
	\label{fig:frequences}
\end{figure}

On remarque bien que pour des valeurs de $ \theta_0 $ faibles représentant les petites oscillations du pendule, 
la fréquence semble bien égale à la valeur $ \frac{1}{2 \pi} \sqrt{\frac{g}{l}} $.

\subsection{Double pendule}
Le pendule à deux maillons est constitué de deux masses $ m_1 $ et $ m_2 $ 
reliées par deux tiges de longueur $ l_1 $ et $ l _2 $. On note $ \theta_1 $ et $ \theta_2 $ les angles
que font les tiges avec la verticale. Dans la littérature, on trouve les deux équations donnant $\ddot \theta_1$ et $\ddot \theta_2$ en fonction des autres paramètres.

La mise sous forme du système d'équations différentielles comme précisé en première partie est plus complexe,
puisqu'il faut passer par un vecteur de résolution de dimension $4$.
En effet, le système possède $4$ conditions initiales, à savoir les angles initiaux, et les vitesses initiales.
On a donc le vecteur $ U = (\theta_1, \theta_2, \dot \theta_1, \dot \theta_2) $, et on peut donc écrire sa dérivée en fonction de ses variables, et des équations donnant $\ddot \theta_1$ et $\ddot \theta_2$,
ce qui nous donne le vecteur $ U' = (\dot \theta_1, \dot \theta_2, \ddot \theta_1, \ddot \theta_2) $.

À l'inverse du pendule simple où les trajectoires sont des ellipses peu importe les conditions initiales, 
le pendule double est quant à lui un système très sensible à celles-ci: c'est un système chaotique.  

\medbreak

Une fois les trajectoires calculées, on peut s'intéresser au temps que met le pendule à se retourner (la masse $ m_2 $ passe au
dessus de $ m_1 $).
Ce retournement arrive lorsque $ \theta_2 $ dépasse en valeur absolue $ \pi $.
Pour des valeurs faibles des conditions initiales, on peut observer que le pendule ne se retourne pas, comme dans la figure~\ref{fig:no_retournement}.
En revanche, lorsqu'on augmente les vitesse angulaires, on peut généralement observer un retournement du pendule, comme dans la figure~\ref{fig:retournement}.
Ces deux figures ont été tracées avec des conditions initiales nulles, excepté pour la vitesse angulaire $ \dot \theta_2 $.

\begin{figure} [htbp!]
	\begin{minipage}[c]{0.45\textwidth}
		\centering
		\includegraphics[width=\textwidth]{res/no_retournement.png}
		\caption{Pas de retournement, $\dot \theta_2 = 4$ rad}
		\label{fig:no_retournement}
	\end{minipage}\hfill
	\begin{minipage}[c]{0.45\textwidth}
		\centering
		\includegraphics[width=\textwidth]{res/retournement.png}
		\caption{Retournement, $\dot \theta_2 = 8$ rad}
		\label{fig:retournement}
	\end{minipage}
\end{figure}
