\documentclass{article}

\usepackage[english]{babel}
\usepackage[utf8]{inputenc}
\usepackage[]{hyperref}
\usepackage{graphicx, float}
\usepackage{subfig}
\usepackage{amssymb, amsmath, amsthm, stmaryrd}

%%%%%%%%%%%%%%%% Lengths %%%%%%%%%%%%%%%%
\setlength{\textwidth}{15.5cm}
\setlength{\evensidemargin}{0.5cm}
\setlength{\oddsidemargin}{0.5cm}

%%%%%%%%%%%%%%%% Variables %%%%%%%%%%%%%%%%
\def\projet{6}
\def\titre{Résolution approchée d'équations différentielles ordinaires}
\def\groupe{4}
\def\equipe{4}
\def\responsible{Antton CATTARIN}
\def\secretary{Mathieu MOREL}
\def\others{Arthur LE FLOCH, Manoa BEAUVAIS}

\begin{document}

%%%%%%%%%%%%%%%% Header %%%%%%%%%%%%%%%%
\noindent\begin{minipage}{0.98\textwidth}
	\vskip 0mm
	\noindent
	{ \begin{tabular}{p{7.5cm}}
		{\bfseries \sffamily
			Projet \projet} \\ 
		{\itshape \titre}
		\end{tabular}}
	\hfill 
	\fbox{\begin{tabular}{l}
		{~\hfill \bfseries \sffamily Groupe \groupe\ - \'Equipe \equipe
			\hfill~} \\[2mm] 
		Responsable : \responsible \\
		Secrétaire : \secretary \\
		Codeurs : \others
		\end{tabular}}
	\vskip 4mm ~

	~~~\parbox{0.95\textwidth}{\small \textit{Résumé~:} \sffamily
		Le but de ce projet est d'implémenter des algorithmes de résolution approchée d'équations différentielles ordinaires,
		et d'en évaluer les performances.
		Ce rapport traitera de l'application à la résolution du problème du système proie-prédateur de Lotka-Volterra, ainsi que
		le problème du double pendule.
	}
	\vskip 1mm ~
\end{minipage}

%%%%%%%%%%%%%%%% Main part %%%%%%%%%%%%%%%%

\section{Résolution d'équations différentielles ordinaires}

Dans cette partie, on cherche à programmer différentes méthodes de résolution
des équations différentielles ordinaires.
Ces algorithmes utilisent tous des méthodes avec un pas de discrétisation, de la forme
$y_{n+1}=y_n+h_n \times \phi (y_n,t_n,h_n)$ .

\subsection{Représentation d'un problème de Cauchy}
Tout d'abord, on cherche à définir la façon dont on va représenter
un problème de Cauchy dans notre code. Un problème de Cauchy est défini par :
\begin{equation}
	\begin{cases}
		y(t_0)=y_0 ~\text{(condition initiale)}\\
		y’(t)=f(y(t),t) \\
	\end{cases}    
\end{equation}
Le vecteur $y_0$ représente les conditions initiales, et est modélisé en pratique par un tableau de $n$ éléments.
La fonction $y'$ donne la dérivée par rapport au temps, en fonction de $y$ et $t$.
Pour résoudre une équation différentielle, nous chercherons donc à mettre le problème sous cette forme.

\subsection{Méthodes de résolution}
Nous avons implémenté les méthodes de résolution suivantes : méthode d’Euler, méthode du point milieu,
méthode de Heun et méthode de Runge-Kutta d’ordre 4. Elles correspondent respectivement aux équations
\ref{eq:euler}, \ref{eq:point-milieu}, \ref{eq:heun} et \ref{eq:runge-kutta}.

\begin{equation}
	y_{n+1} = y_n + h_n \times f(y(t_n), t_n)
	\label{eq:euler}
\end{equation}

\begin{equation}
	y_{n+1} = y_n + h_n \times f \left(y_n + \frac{h_n}{2} \times f(y_n, t_n), t_n + \frac{h_n}{2}\right)
	\label{eq:point-milieu}
\end{equation}

\begin{equation}
		y_{n+1} = y_n + \frac{h_n}{2} \times (f(y_n, t_n) + f(y_n + h_n \times f(y_n, t_n), t_n + h_n))
	\label{eq:heun}
\end{equation}

\begin{equation}
	\begin{cases}
		k_{n, 1} = f(y_n, t_n)\\
		k_{n, 2} = f(y_n + \frac{1}{2} h_n k_{n, 1}, t_n + \frac{1}{2}h_n)\\
		k_{n, 3} = f(y_n + \frac{1}{2} h_n k_{n, 2}, t_n + \frac{1}{2}h_n)\\
		k_{n, 4} = f(y_n + h_n k_{n, 3}, t_n + h_n)\\
		y_{n+1} = y_n + \frac{1}{6} h_n (k_{n, 1} + 2k_{n, 2} + 2k_{n, 3} + k_{n, 4})
	\end{cases}
	\label{eq:runge-kutta}
\end{equation}

On a donc implémenté ces méthodes de la façon la plus générique possible. Tout d'abord, 
chaque fonction correspondant à une méthode est nommée sous la forme : \texttt{step\_method\_name(y,t,h,f)} et calcule
une unique itération de la méthode.

Nous avons ensuite implémenté \texttt{meth\_n\_step} qui calcule un nombre $N$ d'étapes avec un pas constant $h$,
puis \texttt{meth\_epsilon} qui calcule une solution approchée avec un paramètre d’erreur $\epsilon$.
En pratique, à chaque itération si l'erreur entre deux itérations successives est inférieure à $\epsilon$,
alors on arrête le calcul et on renvoie la solution.

Nous avons aussi implémenté une fonction \texttt{meth\_epsilon\_convergence} nous permettant de visualiser
sur un graphique la convergence de nos solutions.
La figure~\ref{fig:subdivision} nous montre ainsi une comparaison des résultats en fonction
du nombre de subdivisions $N$, par la méthode d'Euler.
Comme attendu, on observe une convergence de la solution vers la solution exacte (tracée en noir)
lorsque le nombre de subdivisions augmente.

\begin{figure}[htbp!]
	\centering
	\includegraphics[width=0.5\textwidth]{res/subdivisions}
	\caption{Comparaison la convergence des solutions obtenues en fonction du nombre de subdivisions, pour l'équation différentielle $y'(t) = \frac{1}{1 + t^2}$ avec $y(0) = 1$.}
	\label{fig:subdivision}
\end{figure}

Sur l'ensemble des quatres méthodes de résolution, la plus précise est celle de Runge-Kutta, puisqu'elle est d'ordre $4$.

En revanche, elle est aussi la plus coûteuse en temps de calcul, puisqu'elle nécessite, au minimum, quatres appels à la fonction $f$ pour un unique pas de temps.

La méthode d'Euler est la moins précise (ordre $1$), mais aussi la plus rapide, puisqu'elle ne nécessite qu'un seul appel à la fonction $f$ par pas de temps.

Les autres méthodes constituent un bon compromis entre précision et temps de calcul, puisque la méthode du point milieu et celle de Heun sont d'ordre $2$, et nécessitent deux appels à la fonction $f$ par pas de temps.

\subsection{Champ des tangentes}
Pour continuer, nous avons implémenté la fonction \texttt{tangent\_2D} qui nous a permis de tracer le champ des tangentes
des équations différentielles de dimension $2$.

Comme on peut le voir sur la figure~\ref{fig:tangente}, nous avons tracé cette courbe des tangentes sur un exemple d'équation différentielle
de dimension $2$ sous la forme $y(t)=(y_1(t),y_2(t))$, avec $y(0)=(1,0)$ et $y'(t)=(-y_2(t),y_1(t))$.

Les résultats que nous obtenons pour ces champs de tangentes sont proches des résultats exacts.

\begin{figure}[htbp!]
	\centering
	\includegraphics[width=0.7\textwidth]{res/tangente}
	\caption{Tracé de la solution de l'équation différentielle en fonction du temps, et son champ des tangentes}
	\label{fig:tangente}
\end{figure}


\section{Système proie-prédateur de Lotka-Volterra}
Dans cette partie, nous allons étudier le système proie-prédateur de Lotka-Volterra,
qui est un système d'équations différentielles ordinaires,
et le comparer à des modèles plus simples.


\subsection{Modèles simplistes}
Un des premiers modèle de population est le modèle de Malthus, qui suppose que la population croît de manière exponentielle.
Celui-ci utilise l'équation $\frac{dN(t)}{dt} = \gamma N(t)$, avec $N(t)$ la population à l'instant $t$, et $\gamma \in \mathbb{R}$.

Un coefficient $\gamma$ positif implique une croissance de la population, et un coefficient négatif une décroissance.
Un autre modèle a été proposé, celui de Verhulst, utilisant une formule similaire à celle de Malthus,
à la différence qu'il ne croît pas au delà d'une certaine population $\kappa$, ce qui simule une limite de ressources.
Il utilise l'équation $\frac{dN(t)}{dt} = \gamma N(t) \left( 1 - \frac{N(t)}{\kappa} \right)$,
avec $N(t)$ la population à l'instant $t$ (à un facteur près), $\gamma \in \mathbb{R}$ et $\kappa \in \mathbb{R}$.

Les résultats de ces deux modèles sont présentés sur la figure~\ref{fig:populations}, ce qui montre bien que le modèle de Verhulst
est plus réaliste que celui de Malthus.

\begin{figure}[htbp!]
	\centering
	\includegraphics[width=0.7\textwidth]{res/population}
	\caption{Comparaison des modèles de Malthus et de Verhulst, avec initialement $40$ individus.}
	\label{fig:populations}
\end{figure}

Le modèle de Lotka-Volterra est un modèle plus réaliste, qui prend en compte deux populations,
l'une de proies, l'autre de prédateurs.
Il utilise les équations~\ref{eq:lotka}, avec $N(t)$ la population de proies à l'instant $t$ (à un facteur près),
$P(t)$ la population de prédateurs à l'instant $t$, $(a, b, c, d) \in (\mathbb{R}^{+*})^4$.

\begin{equation}
	\label{eq:lotka}
	\frac{dN(t)}{dt} = N(t) \times (a - b \times P(t)) \ \ \ \ \ \ \ \ \ \ \ \ \ \ \
	\frac{dP(t)}{dt} = P(t) \times (c \times N(t) - d)
\end{equation}

Après résolution avec la méthode de résolution \textit{epsilon}, on obtient la figure~\ref{fig:lotka}, qui montre que les deux populations sont périodiques.

\begin{figure}[htbp!]
	\centering
	\includegraphics[width=0.7\textwidth]{res/lotka}
	\caption{Résolution du système de Lotka-Volterra, avec initialement $2$ proies et $2$ prédateurs.}
	\label{fig:lotka}
\end{figure}

De sorte à calculer rapidement la période des oscillations des deux populations, on itère
sur les valeurs échantillonnées, et on vérifie si la valeur actuelle est proche de la toute première.

Concernant les solutions constantes pour ce modèle, elles sont obtenues pour des dérivées nulles,
ce qui se traduit par la formule logique~\ref{eq:lotka-const}. Ce sont les points singuliers
de l'équation différentielle.

\begin{equation}
	\label{eq:lotka-const}
	\left(\left(N(t) = 0\right) \vee \left(P(t) = \frac{a}{b}\right)\right)
	\wedge
	\left(\left(P(t) = 0\right) \vee \left(N(t) = \frac{d}{c}\right)\right), \forall t \in \mathbb{R}
\end{equation}

En traçant les solutions proches d'un point non-singulier, on obtient la figure~\ref{fig:behaviour}. De manière systématique,
pour des points non-singuliers, on obtient des solutions périodiques, et donc un cycle fermé, de forme déterminée par les paramètres.
Pour des points singuliers, on obtient des ellipses si les valeurs ne sont pas nulles, selon la formule~\ref{eq:lotka-const},
et des droites si l'une des valeurs est nulle, et un point si les deux sont nulles.

\begin{figure}[htbp!]
	\centering
	\includegraphics[width=0.45\textwidth]{res/behaviour}
	\caption{Comportement des solutions proches d'un point non-singulier (rectangle autour de $(2, 2)$)}
	\label{fig:behaviour}
\end{figure}

Dans le cas général, pour une équation différentielle quelconque, il n'y a pas nécessairement de cycle fermé dans une représentation
telle que celle de la figure~\ref{fig:behaviour}.


\section{Pendule à $N$ maillons}

\renewcommand*{\overrightarrow}[1]{\vbox{\halign{##\cr 
  \tiny\rightarrowfill\cr\noalign{\nointerlineskip\vskip1pt} 
  $#1\mskip2mu$\cr}}}

Dans cette partie, nous allons considérer deux types de pendules à $N$ maillons (le pendule simple et le pendule double) et chercher à 
résoudre leurs équations de mouvement par les méthodes numériques implémentées dans la section~\ref{sec:sec1}.

\subsection{Pendule simple}
Dans le cas du pendule simple, il y a deux forces qui s'appliquent sur la masse $m$:
le poids de la masse \overrightarrow{P} et la force de la tige \overrightarrow{T}.
Selon le théorème du moment cinétique appliqué en $M$ par rapport à $O$, on a
$\frac{d\overrightarrow{L_{O,M}}}{dt} = \overrightarrow{M_{O, M}(\overrightarrow{P})} + \overrightarrow{M_{O, M}(\overrightarrow{T})}$.

Le moment de \overrightarrow{T} est nul, car \overrightarrow{OM} et \overrightarrow{T} sont colinéaires. En développant l'expression, on a alors 
$m l^{2} \ddot \theta = - m l g \sin{\theta} $, ce qui nous donne finalement l'équation $\ddot \theta + \frac{g}{l} \sin{\theta}= 0$.

On retrouve l'équation différentielle non linéaire du second ordre bien connue du pendule simple. 
Pour la résoudre numériquement en fonction de l'angle initial $ \theta_0 $, et de la vitesse initiale $ \dot \theta_0 $,
on peut utiliser une des méthodes implémentées auparavant, comme par exemple la méthode d'Euler.
Une fois que la fonction $ \theta(t) $ a été approchée, on peut calculer la période du pendule en prenant la différence entre 
deux maxima successifs, ce qui permet de déterminer la fréquence.

En affichant les différentes fréquences du pendule pour des valeurs de $ \theta_{0} $ variant de $ -\frac{\pi}{2} $ à $ \frac{\pi}{2} $, 
on obtient la figure~\ref{fig:frequences}.

\begin{figure}[htbp!]
	\centering
	\includegraphics[width=0.45\textwidth]{res/freq_pendule_simple.png}
	\caption{Fréquence du pendule en fonction de l'angle initial $ \theta_{0}$}
	\label{fig:frequences}
\end{figure}

On remarque bien que pour des valeurs de $ \theta_0 $ faibles représentant les petites oscillations du pendule, 
la fréquence semble bien égale à la valeur $ \frac{1}{2 \pi} \sqrt{\frac{g}{l}} $.

\subsection{Double pendule}
Le pendule à deux maillons est constitué de deux masses $ m_1 $ et $ m_2 $ 
reliées par deux tiges de longueur $ l_1 $ et $ l _2 $. On note $ \theta_1 $ et $ \theta_2 $ les angles
que font les tiges avec la verticale. Dans la littérature, on trouve les deux équations donnant $\ddot \theta_1$ et $\ddot \theta_2$ en fonction des autres paramètres.

La mise sous forme du système d'équations différentielles comme précisé en première partie est plus complexe,
puisqu'il faut passer par un vecteur de résolution de dimension $4$.
En effet, le système possède $4$ conditions initiales, à savoir les angles initiaux, et les vitesses initiales.
On a donc le vecteur $ U = (\theta_1, \theta_2, \dot \theta_1, \dot \theta_2) $, et on peut donc écrire sa dérivée en fonction de ses variables, et des équations donnant $\ddot \theta_1$ et $\ddot \theta_2$,
ce qui nous donne le vecteur $ U' = (\dot \theta_1, \dot \theta_2, \ddot \theta_1, \ddot \theta_2) $.

À l'inverse du pendule simple où les trajectoires sont des ellipses peu importe les conditions initiales, 
le pendule double est quant à lui un système très sensible à celles-ci: c'est un système chaotique.  

\medbreak

Une fois les trajectoires calculées, on peut s'intéresser au temps que met le pendule à se retourner (la masse $ m_2 $ passe au
dessus de $ m_1 $).
Ce retournement arrive lorsque $ \theta_2 $ dépasse en valeur absolue $ \pi $.
Pour des valeurs faibles des conditions initiales, on peut observer que le pendule ne se retourne pas, comme dans la figure~\ref{fig:no_retournement}.
En revanche, lorsqu'on augmente les vitesse angulaires, on peut généralement observer un retournement du pendule, comme dans la figure~\ref{fig:retournement}.
Ces deux figures ont été tracées avec des conditions initiales nulles, excepté pour la vitesse angulaire $ \dot \theta_2 $.

\begin{figure} [htbp!]
	\begin{minipage}[c]{0.45\textwidth}
		\centering
		\includegraphics[width=\textwidth]{res/no_retournement.png}
		\caption{Pas de retournement, $\dot \theta_2 = 4$ rad}
		\label{fig:no_retournement}
	\end{minipage}\hfill
	\begin{minipage}[c]{0.45\textwidth}
		\centering
		\includegraphics[width=\textwidth]{res/retournement.png}
		\caption{Retournement, $\dot \theta_2 = 8$ rad}
		\label{fig:retournement}
	\end{minipage}
\end{figure}


\end{document}